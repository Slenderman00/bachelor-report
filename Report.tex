% !TeX root = Report.tex
\documentclass[12pt]{article}

% Package imports (organized and deduplicated)
\usepackage{biblatex}
\usepackage{changepage}
\usepackage{color}
\usepackage{enumitem}
\usepackage{float}
\usepackage{graphicx}
\usepackage{listings}
\usepackage{sectsty}
\usepackage{xcolor}
\usepackage[breaklinks=true]{hyperref}
\usepackage{xurl}
\usepackage{tikz}
\usepackage{lipsum}
\usepackage[format=plain,
            labelfont=it,
            textfont=]{caption}
\usetikzlibrary{shapes.geometric,positioning,fit,backgrounds}
\usepackage{./timing-diagrams}
\usetikzlibrary{calc}
\setcounter{biburlnumpenalty}{100}
\setcounter{biburlucpenalty}{100}
\setcounter{biburllcpenalty}{100}

\newcommand{\subsubsubsection}[1]{\paragraph{#1}\mbox{}\\}
\setcounter{secnumdepth}{4}
\setcounter{tocdepth}{4}

\newcommand{\writersnote}[1]{\marginpar{\small{\textcolor{blue}{Writer's note:}} \scriptsize\textit{#1}}}

% \usepackage{background}
% \backgroundsetup{
%   position=current page.north west,
%   angle=0,
%   nodeanchor=north west,
%   vshift=-1cm,
%   hshift=1cm,
%   color=red,
%   opacity=1,
%   scale=1,
%   contents={Preprint}
% }

\definecolor{darkblue}{RGB}{0, 0, 102} 
\hypersetup{
    colorlinks=true,
    pdfborder={0 0 0},
    linkbordercolor=white,
    urlcolor=darkblue,
    linkcolor=darkblue,
    citecolor=darkblue,
    filecolor=darkblue
}

% Make bibliography ragged right instead of justified
\AtBeginDocument{
  \renewcommand{\bibsetup}{\raggedright}
}
% Document configuration
\restylefloat{table}
\graphicspath{{./images/}}
\addbibresource{Library.bib}
\subsectionfont{\fontsize{12}{14}\selectfont}

% Author information
\author{
    Group Number: 107\\
    Joar Heimonen, Christian Vu, Naly Keli \\
    \texttt{contact@joar.me}\\ 
    \texttt{chvu002@student.kristiania.no}\\
    \texttt{nake002@student.kristiania.no}
}

% Title configuration
\title{
  \textbf{Preliminary Title}\\
  \large{Preliminary Description}
}
\date{\today}

\newcommand{\license}{
    \vspace{1em}
    \noindent\small{© 2025 Joar Heimonen\\
    This work is licensed under a \href{https://creativecommons.org/licenses/by-sa/4.0/}{Creative Commons Attribution-Sharealike 4.0 International License}.}
}
\begin{document}
\maketitle
\pagebreak

\tableofcontents

\pagebreak


\section{Introduction}
There are many paradigms of commercial sensor management and monitoring. Organizations can use anything from 
PLC (programmable Logic Controllers) to IoT devices to manage and monitor their sensors. For commercial use 
some of these alternatives are more popular than others. There are also a large amount of different higher level protocols
like MQTT, HTTP and SNMP that can be used to manage and monitor sensors. We propose using the NETCONF protocol 
with YANG sensor models for management. This work will be done in collaboration with Lightside Instruments AS.

This document will cover the following three topics:
\begin{itemize}
  \item \textbf{Work methodology:} An indept analysis of the knowledge base around work methods like Scrum, Kanban, and Waterfall. 
  With a focus on how our work methodology differs from these.
  \item \textbf{NETCONF and YANG sensor management}: A qualitative analysis of the NETCONF and YANG protocols and how they can be used to manage sensors.
  \item \textbf{NETCONF Security}: A qualitative analysis of the security aspects of the NETCONF protocol.
\end{itemize}

\section{Lightside Instruments AS}
Lightside Instruments is a company specializing in developing instruments with model based network management 
for use in networking, network interconnect testing and telemetry. 
They design their instruments with YANG (RFC7950) models and NETCONF (RFC6241) \cite{ennsNetworkConfigurationProtocol2011} protocol. 
The instruments are based on IETF standards and drafts, 
and are implemented with software tools available in Debian, programmable 
logic and open hardware. \cite{lsi}

\section{Technical background}

\subsection{NETCONF and YANG}
NETCONF \cite{ennsNetworkConfigurationProtocol2011} is a model based Network Configuration Protocol.
Each NETCONF device presents the aquiring device with a YANG \cite{bjorklundYANG11Data2016} data model
consisting of the device state and parameters. 
Each data model has a set of constraints making them error correcting.

\section{Work methodology}

\section{NETCONF and YANG sensor management}

\subsection{Node RED}
\subsubsection{node-yuma123}
\subsubsubsection{easyNetconf}

\section{NETCONF Security}

\pagebreak
\addcontentsline{toc}{section}{References}
\printbibliography
\license
\end{document}
